\documentclass[12pt,reqno]{amsart}
\usepackage{graphicx}
\usepackage{amsfonts,amssymb,latexsym,amsmath}
\usepackage{tikz-cd}
\usepackage{mathrsfs}
\usepackage{enumerate}
\usepackage[cm]{fullpage}
\theoremstyle{plain}
\newtheorem*{theorem*}{Theorem}
%% this allows for theorems which are not automatically numbered
\newtheorem{defi}{Definition}
\newtheorem{theorem}{Theorem}
\newtheorem{lemma}{Lemma}
\newtheorem{rem}{Remark}
\newtheorem{disc}{Discussion}
\newtheorem{prop}{Proposition}
\newtheorem{coro}{Corollary}
\newtheorem{prob}{Problem}
\newtheorem{conv}{Convention}
\newtheorem{layout}{Layout}
\newtheorem{ex}{Example}
\usepackage{lineno}
\usepackage{hyperref}
\usepackage{xcolor}
\definecolor{winered}{rgb}{0.5,0,0}
\hypersetup{
    colorlinks=true,
    linkcolor=blue,
    filecolor=magenta,      
    urlcolor=winered,
}
%% The above lines are for formatting.  In general, you will not want to change these.
%%Commands to make life easier
\newcommand{\RR}{\mathbf R}
\newcommand{\CC}{\mathbf C}
\newcommand{\oo}{\mathcal{O}}
\newcommand{\ZZ}{\mathbf Z}
\newcommand{\ZZn}[1]{\ZZ/{#1}\ZZ}
\newcommand{\QQ}{\mathbf Q}
\newcommand{\rr}{\mathbb R}
\newcommand{\cc}{\mathbb C}
\newcommand{\nn}{\mathbb N}
\newcommand{\zz}{\mathbb Z}
\newcommand{\zzn}[1]{\zz/{#1}\zz}
\newcommand{\qq}{\mathbb Q}
\newcommand{\calM}{\mathcal M}
\newcommand{\latex}{\LaTeX}
\newcommand{\tex}{\TeX}
\newcommand{\sm}{\setminus} 
\newcommand{\dom}{\text{Dom}}
\DeclareMathOperator{\Hom}{Hom}
\newcommand{\sym}{\textup{Sym}}
\newcommand{\ran}{\text{Ran}}
\newcommand{\pp}{\prime}
\newcommand{\gap}{\; \; \;}
\newcommand{\Mod}[1]{\ (\mathrm{mod}\ #1)}
\frenchspacing
\usepackage[shortlabels]{enumitem}
\usepackage{mdframed}
\usepackage{lipsum}
\def\boxitem#1{\setbox0=\vbox{#1}{\centering\makebox[0pt]{%
  \fboxrule=.5pt\color{black}\fbox{\hspace{\leftmargini}\color{black}\box0}}\par}}



\title{Tensors: Notes on related material}
\begin{document}
\maketitle
\tableofcontents
\section{Motivation}

An introduction to learning about the tensor product and \textit{pure} tensors themselves are often times, in my experience, difficult to comprehend — or at least daunting. For example, if you're to pickup Eisenbud's \textit{Commutative Algebra with a View Toward Algebraic Geometry} you'll encounter the tensor product quickly enough and if you don't know already what it \textit{is} and \textit{does}, you might go the appendix and try to comprehend the definition. 
\begin{defi}[Eisenbud] Let $M$ and $N$ be $R$-modules. The \textbf{tensor product} of $M$ and $N$ over $R$, written $M \otimes_R N$, is the $R$-module generated by symbols $m \otimes n$ for $m \in M$ and $n \in N$, with relations 
\begin{align*}
rm \otimes n &= m \otimes rn \\
(m + m^{\prime}) \otimes n &= m \otimes n + m^{\prime} \otimes n  \\
m \otimes (n + n^{\prime}) &= m \otimes n + m \otimes n^{\prime}.
\end{align*}
\end{defi} 

Eisenbud then states, immediately, that ``these relations say precisely that the natural map 
\\ $b \colon M \times N \to M \otimes_R N$ taking $(m, n)$ to $m \otimes n$ is \textbf{bilinear}." Now, as for me, I immediately thought ``What the fuck is bilinear?" I had taken a computationally focused linear algebra course in high school so the theoretical ideas didn't really stick, and so it took some formal linear algebra enlightenment to further understand what Eisenbud was talking about. Really what this exposition is all about is an introduction, or refreshment, of the formal linear algebra ideas that mathematicians seems to always bring up — you simply cannot avoid linear algebra in any of the natural science, nor mathematics (this is a topic that is milked so much).  

Now, another aspect of this exposition that may be useful is the categorical foundations that I may get into since they're as well used so often in say commutative algebra, homological algebra, algebraic geometry, K-theory, algebraic topology, and by god this list can likely stretch so far out. And well you probably suspected it, they're also integral in understanding different perspectives (or you may argue the main perspective) of the tensor product. For example, as Eisenbud continues talking about the tensor product in his appendix (Appendix A2. Multilinear Algebra): ``It is elementary that a bilinear map $M \times N \to P$ is the same as a homomorphism $M \to \Hom_R (N, P)$, so we may rewrite the natural isomorphism above as a natural isomorphism 
$$ \Hom_R (M \otimes_R N, P) \cong \Hom_R (M, \Hom_R (N, P)).$$
In categorical language this says that the functor $- \otimes_R N$ is the left adjoint functor of the functor $\Hom_R (N,  -)$. "

Although I've mentioned the case of approaching Eisenbud's tensor product definition, there are some other examples that I think are fruitful in mentioning since I'll be trying to really, in a way, comprehend them (myself included) in later sections. 
\newpage
One that comes to mind is provided in Joseph Rotman's \textit{An Introduction to Homological Algebra}. This definition, however, requires more machinery; first we need to describe what we mean by bilinear, and thus we also need to know what biadditive is.  This is all in having the idea of describing the tensor product as a biadditive function between modules, or in the \textit{special case}, between vector spaces. 
\begin{defi}[Rotman] Let $R$ be a ring , let $A_R$ be a right $R$-module, let $_R B$ be a left $R$-module, and let $G$ be an (additive) abelian group. A function $f \colon A \times B \to G$ is called $R$-\textbf{additive} if, for all $a, a^{\prime} \in A, b, b^{\prime} \in B$, and $r \in R$, we have 
\begin{align*}
f(a+a^{\prime}, b) &= f(a, b) + f(a^{\prime}, b), \\
f(a, b+b^{\prime}) &= f(a, b)+ f(a, b^{\prime}), \\
f(ar, b) &= f(a, rb).
\end{align*}

If $R$ is \textit{commutative} and $A$, $B$, and $M$ are $R$-modules, then a function $f \colon A \times B \to M$ is called $R$-\textbf{bilinear} if $f$ is $R$-biadditive and also
$$f (ar, b) = f(a, rb) = r f(a, b)$$
$[rf(a, b)$ makes sense here because $f(a, b)$ now lies in the $R$-module $M]$. 
\end{defi} 

A clear example, that Rotman provides,  is that if we have a ring $R$, then we will get, naturally, that its multiplication $\varphi \colon R \times R \to R$ is $R$-additive — as Rotman says, ``the first two axioms are the right and left distributive laws, while the third axiom is associativity." You can check these easily on your own.  Now, since we didn't assume $R$ to be commutative, we can then suppose that it is and then we'll get that $\varphi$ is bilinear since $\varphi \colon R \times R \to R$ is biadditive and $\varphi (ar, b) = (ar)b= a(rb) = \varphi (a, rb) = r(ab) = r \varphi (a, b)$ also holds. 

The purpose of describing all this machinery is to simply explain the tensor product as a sort of conversion process between biadditive functions and linear functions.
\begin{rem}[Rotman]The tensor product converts biadditive functions into linear ones. 
\end{rem} 
\begin{defi}[Rotman] Given a ring $R$ and modules $A_R$ and $_R B$, then their \textbf{tensor product}  is an abelian group $A \otimes_R B$ and an $R$-biadditive function
$$ h \colon A \times B \to A \otimes_R B$$ such that, for every abelian group $G$ and every $R$-biadditive $f \colon A \times B \to G$, there exists a unique $\zz$-homomorphism $\tilde{f} \colon A \otimes_R B \to G$ making the following diagram commute. 
$$\begin{tikzcd}
A \times B \arrow[rrrr, "h"] \arrow[rrdd, "f"'] &  &   &  & A \otimes_R B \arrow[lldd, "\tilde{f}", dashed] \\
                                       &  &   &  &                             \\
                                       &  & G &  &                            
\end{tikzcd}$$
\end{defi} 

If you don't already know what $\zz$-homomorphism is, a simple google search will provide, I'm sure,  a decent explanation. Moreover, if can't recall immediately what a linear function is, as it's important to know for Remark 1:
\begin{defi} A \textbf{linear function} is a mapping $\lambda \colon A \to M$ between $R$-modules $A$ and $M$ such that it satisfies the properties
\begin{align*}
\lambda (a+b) &= \lambda (a) + \lambda (b) \\
\lambda (\alpha b) &= \alpha \lambda (b)
\end{align*}
for all $a, b \in A$ and $\alpha \in R$.
\end{defi} 
\newpage 
\section{Module Theory}

To establish some common notation, and provide a general refresher, we'll go through an introduction to the ideas of module theory. If this topic is unknown to you, it's a \textit{weaker} theory of vector spaces: a vector space takes its scalars from a field (e.g., $\rr$, $\cc$, $\qq$, etc...), while a module takes its scalars from a ring such as the integers. 
\end{document}